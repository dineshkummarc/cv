%!TEX TS-program = xelatex
\documentclass[]{friggeri-cv}

\usepackage{fancyhdr}

% Add page numbering in the bottom-right corner.
\pagestyle{fancy}
\fancyfoot{}
\fancyfoot[R]{\footnotesize \thepage\ / 2}
\renewcommand{\headrulewidth}{0pt}
\renewcommand{\footrulewidth}{0pt}

\colorlet{lightergray}{lightgray!40}

\begin{document}

\header{merwan}{achibet}
       {}

% In the aside, each new line forces a line break
\begin{aside}
  \section{Contact}
    ~
    31 rue Toulouse Lautrec
    76620 Le Havre
    France
    ~
    achibetmerwan{}@gmail.com
    ~
    06.03.30.56.77
    \section{Web}
    ~
    \href{http://github.com/merwaaan}{github.com/merwaaan}
    \href{http://merwaaan.github.com}{merwaaan.github.com}
    ~
    {\color{lightergray}\rule{3cm}{0.01cm}}
    ~
    22 ans
    Détenteur du permis B
\end{aside}

\section{Interêts}

Simulation de phénomènes du réel, intelligence artificielle, web
dynamique, systèmes complexes, graphes et réseaux d'interaction,
mondes virtuels.

\section{Formation}

\begin{entrylist}
  \entry
    {depuis 2010}
    {Master Informatique}
    {Université du Havre}
    {Modélisation, Interactions et Systèmes Complexes.}
  \entry
    {2007–2010}
    {Licence Mathématiques-Informatique}
    {Université du Havre}
    {}
  \entry
    {2006–2007}
    {Baccalauréat Scientifique}
    {Lycée Jules Siegfried, Le Havre}
    {Specialités mathématiques et sciences de l'ingénieur.}
\end{entrylist}

\section{Connaissances}

Modélisation des systèmes complexes, intelligence artificielle, web
dynamique, graphe, infographie et OpenGL, informatique distribuée,
algorithmique, programmation orientée objet, programmation
fonctionnelle, programmation logique, architecture, base de données,
compilation, interface homme-machine, optimisation combinatoire,
informatique théorique, réseau, systèmes différentiels, analyse
fonctionnelle.

\section{Expérience}

\begin{entrylist}
  \entry
    {actuellement}
    {LITIS/Université du Havre}
    {Stage de recherche}
    {Coévolution du viaire et du bâti dans un réseau urbain.}
  \entry
    {2010–2012}
    {Université du Havre}
    {Contrat ponctuel}
    {Représentant de la branche scientifique de l'Université du Havre
      lors de salons étudiants.}
  \entry
    {2010–2011}
    {Université du Havre}
    {Tutorat}
    {Encadrement d'étudiants de licence pour l'étude et la pratique du
    C.}
  \entry
    {06–07 2010}
    {BNP Paribas, Paris}
    {Job d'été}
    {Continuité du stage.}
  \entry
    {04–05 2010}
    {BNP Paribas, Paris}
    {Stage}
    {Réalisation d'un outil de gestion d'expertises immobilières.}
  \entry
    {07–08 2009}
    {Groupama Transport, Le Havre}
    {Job d'été}
    {Maintenance, dépannage et manutention informatique.}
  \entry
    {07–08 2008}
    {Groupama Transport, Le Havre}
    {Job d'été}
    {Maintenance, dépannage et manutention informatique.}
\end{entrylist}

\newpage

\section{Langues}

\begin{entrylist}
  \entry
    {}
    {Français}
    {}
    {Langue maternelle.}
  \entry
    {}
    {Anglais}
    {}
    {Courant, score de 940/990 au TOEIC.}
\end{entrylist}

\section{Autres langues}

\begin{entrylist}
  \entry
    {}
    {Impératives}
    {}
    {C, Java, C++.}
  \entry
    {}
    {Fonctionnelles}
    {}
    {Common Lisp, OCaml.}
  \entry
    {}
    {Logique}
    {}
    {Prolog.}
  \entry
    {}
    {Orientées web}
    {}
    {Javascript, Coffeescript, HTML5, CSS3, PHP, SQL.}
\end{entrylist}

\section{Travaux}

\begin{entrylist}
  \entry
    {2012}
    {Sparkets}
    {\href{http://github.com/fmdkdd/sparkets}{github.com/fmdkdd/sparkets}}
    {Jeu vidéo multijoueur temps-réel par navigateur tirant partie des
      dernières technologies du web : HTML5, Node.js et Socket.IO.}
  \entry
    {2012}
    {Recherche décentralisée de connexité}
    {\href{https://github.com/merwaaan/papers/blob/master/connexite/rapport.pdf?raw=true}{github.com/merwaaan/papers/connexite}}
    {Comment assurer la connexité d'un réseau de capteurs volants
      mobiles dont la distance de communication est limitée tout en
      optimisant la superficie occupée~? Une approche inspirée des
      boïds de Craig Reynolds et des mouvements particulaires est
      proposée.}
  \entry
    {2012}
    {Chaînes de Markov : D'arthémis à Zeus}
    {\href{https://github.com/merwaaan/papers/blob/master/markov/rapport.pdf?raw=true}{github.com/merwaaan/papers/markov}}
    {\'Etude et présentation des chaînes de Markov avec exemple
      applicatif sous la forme d'un programme de génération de noms
      adoptant les mêmes caractéristiques lexicales que ceux des
      protagonistes de la mythologie grecque.}
  \entry
    {2011}
    {Simulation physique de corps rigide avec interaction}
    {\href{http://github.com/merwaaan/physics}{github.com/merwaaan/physics}}
    {Projet de master étudiant la simulation de l'évolution des
      propriétés mécaniques d'objets rigides dans un environnement 3D
      en prenant en compte les interactions collisionnelles s'opérant
      entre les corps et s'étant conclu par la mise au point d'un
      moteur physique.}
  \entry
    {2010}
    {La programmation et l'art}
    {\href{http://github.com/merwaaan/papers/blob/master/art/progart.pdf?raw=true}{github.com/merwaaan/papers/art}}
    {Dissertion réalisée en licence à propos de la relation
      art/programmation ou comment un concept intrinsèquement humain
      peut être appliqué à un support purement synthétique.}

  \entry
    {2010}
    {L'achitecture du Nintendo Entertainment System}
    {\href{https://github.com/merwaaan/papers/blob/master/nes/archines.pdf?raw=true}{github.com/merwaaan/papers/nes}}
    {Projet de licence décrivant l'architecture et les différents
      modules de la console NES ou l'occasion de contempler sous un
      \oe il nouveau et averti les mécaniques sous-jacentes d'un objet
      de mon enfance.}

\end{entrylist}

\end{document}
